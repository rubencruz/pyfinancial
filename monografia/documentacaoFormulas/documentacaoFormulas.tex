\documentclass[a4paper,10pt]{article}


%opening
\title{Documentacao das Formulas}
\author{PyFinancial}

\begin{document}

\maketitle

\begin{abstract}

\end{abstract}

\section{Formulas}


\begin{tabular}{|l|l|l|l|}

\hline
pv - BEG & $ pv = (i+1)^{-n} * ( -fv*i - (i+1) * ( (i+1)^{n} -1)*pmt  ) / i $ & http://www.arachnoid.com/lutusp/finance.html \\ 
\hline
pv - END & $ pv = (i+1)^{-n} * ( -pmt*(i+1)^{n} - fv*i + pmt) / i  $ & http://www.arachnoid.com/lutusp/finance.html \\
\hline
pv - i = 0 & $ pv = fv + n * pmt $ & Material de Camilo e Livro \\
\hline
fv - BEG & $ fv = ( (i+1)*pmt - (i+1)^{n}*(i*pmt + pmt + i*pv) ) / i $ & http://www.arachnoid.com/lutusp/finance.html \\ 
\hline
fv - END & $ fv = ( pmt - (i+1)^{n} * (pmt + i*pv) ) / i $ & http://www.arachnoid.com/lutusp/finance.html \\
\hline
fv - i = 0 & $ fv = - (pv + n*pmt)$ & Material de Camilo e Livro \\  
\hline
n - BEG & $ n = log( (-fv*i + pmt*i + pmt) / (i*pmt + pmt + i*pv) ) / log(i+1) $  & http://www.arachnoid.com/lutusp/finance.html \\ 
\hline
n - END & $ n = log( (pmt - fv*i) / (pmt + i*pv) ) / log(i+1) $ & http://www.arachnoid.com/lutusp/finance.html \\ 
\hline
n - i = 0 & Se pólos com sinal igual: $ n = |(pv - fv)| / |pmt| $, c.c  $ n = (|pv| - |fv|) / |pmt|   ^{1} $. & Material de Camilo e Livro \\
\hline
pmt - BEG & $ pmt = - i*( pv* ( i+1 )^{n} + fv ) / ( (i+1)*( (i+1)^{n} - 1 ) )$ & http://www.arachnoid.com/lutusp/finance.html \\ 
\hline
pmt - END & $ pmt = - i*( pv*(i+1)^{n} + fv ) / ((i+1)^{n} - 1) $ & http://www.arachnoid.com/lutusp/finance.html \\
\hline
pmt - i = 0 & Se pólos com sinal igual: $ pmt = |(pv - fv)| / |n| $, c.c $ pmt = (|pv| - |fv|) / |n|  ^{1} $ & Material de Camilo e Livro \\  
\hline
i & $ i = (|fv / pv|^{1/n} - 1) * 100 $ & http://www.crd2000.com.br/crd012d.htm \\ 
\hline
npv & $ NPV = CF_{0} + CF_{1} / (1+i)^{1} + CF_{2} / (1+i)^{2} + ... + CF_{n} / (1+i)^{n}  $ & Manual da HP c00363319 \\ 
\hline
irr & Resolvido por iteração da fórmula acima até que $ NPV = 0. $ & Livro que está com everton \\
\hline
SAF: pmt & $ pmt = pv * (1+i)^{n} * i / ((1+i)^{n}-1) $ & Material Adail \\
\hline
SAF: amort & $ A_{t} = (pmt - (pv*i)) * (i+1)^{t-1} $ & Material Adail \\
\hline
SAC: juros & $ J_{t} = pv*i - (A_{t}*i*t-1) $ & Material Adail \\
\hline
SAC: pmt & $ pmt_{t} = A_{t} + J_{t} $ & Material Adail \\
\hline
SAC: amort & $ A_{t} = pv / n $ & Material Adail \\
\hline
Conversão do n & $ n_{mensal} = n_{anual} * 12 $ & Manual da HP \\
\hline
Conversão do i (juros simples) & $ i_{mensal} = i_{anual} / 12 $ & Material de Camilo de taxas equivalentes \\
\hline
Conversão do i (juros compostos) & $ i_{mensal} = (1+i_{anual})^{1/12} - 1 $ & Material de Camilo de taxas equivalentes \\

\end{tabular}

Observacoes: 

$ ^{1} $ : Faz-se ainda um novo cálculo do pv com o valor resultante do n. Se o valor retornado for diferente, inverte-se o sinal do n.

\end{document}
