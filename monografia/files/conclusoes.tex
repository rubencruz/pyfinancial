\chapter{Conclusões e Trabalhos Futuros}

Um dos méritos desse projeto foi trabalhar com conhecimento multidisciplinar. Ele envolveu 
conhecimentos adquiridos em disciplinas de computação, como Engenharia de Software, Métodos de
Software Numérico e Sistemas de Informação um e dois, assim como em disciplinas extras, como
Princípios de Administração Financeira. Dessa forma, os integrantes da equipe tiveram a
oportunidade de exercitar o conhecimento adquiridos em sala de aula.

A opção por desenvolver uma aplicação para dispositivos móveis também contribuiu no amadurecimento
da equipe. Pode-se perceber as particularidades e dificuladades de se trabalhar com esse tipo
de dispositivos. Além disso, essa experiência possibilitou o contato com diversas novas tecnologias,
aumentando o conhecimento da equipe e inserindo diferenciais nos currículos.

Uma das dificuldades enfrentadas pelo grupo foi a falta de apoio na área financeira. O cliente
cumpriu muito bem o seu papel, porém ele tinha maior embasamento na área de dispositivos móveis.
Houve diversas tentativas por parte do grupo e do cliente de obter um apoio na área financeira,
porém sem sucesso.

Para trabalhos futuros, pode-se ter a adição de novas fórmulas financeiras, bem como o refatoramento
das fórmulas existentes para introduzir métodos melhores para calcular os resultados. Deve-se
também trabalhar em melhorias na interface com o usuário, visto que isso não foi possível no
decorrer das disciplinas pois a biblioteca gráfica utilizada ainda estava em processo de migração
para a plataforma Nokia Maemo.

Enfim, toda a experiência vivida em Projeto I e II possibilitou o crescimento dos alunos como
desenvolvedores, analistas de projeto e gerentes de projeto. Também aguçou a habilidade de resolução
de problemas e permitiu o contato com novos conceitos e paradigmas.