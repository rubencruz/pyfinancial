\documentclass[a4paper,10pt]{article}


%opening
\title{Documentacao das Formulas}
\author{PyFinancial}

\begin{document}

\maketitle

\begin{abstract}

\end{abstract}

\section{Formulas}

Aqui serão apresentadas as fórmulas usadas, bem como suas origens: \\

\begin{enumerate}
 \item pv - BEG:

\begin{eqnarray*}
pv &=& (i+1)^{-n} * ( -fv*i - (i+1) * ( (i+1)^{n} -1)*pmt  ) / i\\
\end{eqnarray*}

Fonte: http://www.arachnoid.com/lutusp/finance.html

\item pv - END:

\begin{eqnarray*}
	pv &=& (i+1)^{-n} * ( -pmt*(i+1)^{n} - fv*i + pmt) / i \\
\end{eqnarray*}

 Fonte: http://www.arachnoid.com/lutusp/finance.html

\item pv com i = 0: 

\begin{eqnarray*}
  pv &=& fv + n * pmt  \\
\end{eqnarray*}

Fonte: Material de Camilo e Livro Matemática Financeira de João Carlos dos Santos

\item fv - BEG :
\begin{eqnarray*}
 fv &=& ( (i+1)*pmt - (i+1)^{n}*(i*pmt + pmt + i*pv) ) / i \\
\end{eqnarray*}

 Fonte: http://www.arachnoid.com/lutusp/finance.html

\item fv - END: 
\begin{eqnarray*}
fv &=& ( pmt - (i+1)^{n} * (pmt + i*pv) ) / i \\
\end{eqnarray*}

 Fonte: http://www.arachnoid.com/lutusp/finance.html 

\item fv com i = 0:
\begin{eqnarray*}
 fv &=& - (pv + n*pmt) \\
\end{eqnarray*}
 
 Fonte: Material de Camilo e Livro Matemática Financeira de João Carlos dos Santos 

\item  n - BEG:
\begin{eqnarray*}
 n &=& log( (-fv*i + pmt*i + pmt) / (i*pmt + pmt + i*pv) ) / log(i+1) \\
\end{eqnarray*}

  Fonte: http://www.arachnoid.com/lutusp/finance.html 

\item  n - END: 
\begin{eqnarray*}
 n &=& log( (pmt - fv*i) / (pmt + i*pv) ) / log(i+1) \\
\end{eqnarray*}
  
 Fonte: http://www.arachnoid.com/lutusp/finance.html

\item  n com i = 0: 

\begin{itemize}
 \item Se pólos com sinal igual:
	\begin{eqnarray*}
 		 n &=& |(pv - fv)| / |pmt| \\ 		
	\end{eqnarray*}
  \item c.c:
	\begin{eqnarray*}
 		n &=& (|pv| - |fv|) / |pmt|   ^{1} \\	 
	\end{eqnarray*}
\end{itemize}
 
 Fonte: Material de Camilo e Livro Matemática Financeira de João Carlos dos Santos 

\item  pmt - BEG 
\begin{eqnarray*}
	pmt &=& - i*( pv* ( i+1 )^{n} + fv ) / ( (i+1)*( (i+1)^{n} - 1 ) ) \\
\end{eqnarray*}
 
 Fonte: http://www.arachnoid.com/lutusp/finance.html \\ 

\item  pmt - END:
\begin{eqnarray*}
	pmt &=& - i*( pv*(i+1)^{n} + fv ) / ((i+1)^{n} - 1) \\	
\end{eqnarray*}
 
 Fonte: http://www.arachnoid.com/lutusp/finance.html \\

\item  pmt com i = 0:  

\begin{itemize}
 \item Se pólos com sinal igual:
	\begin{eqnarray*}
 		 pmt &=& |(pv - fv)| / |n| \\	
	\end{eqnarray*}
  \item c.c:
	\begin{eqnarray*}
 		pmt &=& (|pv| - |fv|) / |n|  ^{1} \\	 
	\end{eqnarray*}
\end{itemize}

 Fonte: Material de Camilo e Livro Matemática Financeira de João Carlos dos Santos   

\item  i: Usa-se a função do fv com estimativas de i $ ^{2} $ 

 Fonte: http://vps.arachnoid.com/finance/

\item  npv:
\begin{eqnarray*}
 	NPV &=& CF_{0} + CF_{1} / (1+i)^{1} + CF_{2} / (1+i)^{2} + ... + CF_{n} / (1+i)^{n} \\
\end{eqnarray*}
 
 Fonte: Manual da HP c00363319

\item  irr: Resolvido por iteração da fórmula acima até que $ NPV = 0. $ 

 Fonte: Matemática Financeira de Samuel Hazzan e José Nicolau Pompeo

\item SAF: pmt 
\begin{eqnarray*}
 	pmt &=& pv * (1+i)^{n} * i / ((1+i)^{n}-1) \\
\end{eqnarray*}
 
 Fonte: Material Adail

\item  SAF: amort 
\begin{eqnarray*}
 	A_{t} &=& (pmt - (pv*i)) * (i+1)^{t-1} \\	
\end{eqnarray*}
 
 Fonte: Material Adail

\item  SAC: juros 
\begin{eqnarray*}
 	J_{t} &=& pv*i - (A_{t}*i*t-1)	
\end{eqnarray*}
 
 Fonte: Material Adail 

\item  SAC: pmt
\begin{eqnarray*}
 	pmt_{t} &=& A_{t} + J_{t}	\\
\end{eqnarray*}
 
 Fonte: Material Adail

\item  SAC: amort 

\begin{eqnarray*}
 	A_{t} &=& pv / n \\	
\end{eqnarray*}
 
 Fonte: Material Adail

\item  Conversão do n:
\begin{eqnarray*}
 	n_{mensal} &=& n_{anual} * 12 \\	
\end{eqnarray*}
 
 Fonte: Manual da HP Platinum em Português

\item  Conversão do i (juros simples):
\begin{eqnarray*}
 	i_{mensal} &=& i_{anual} / 12 \\	
\end{eqnarray*}
  
 Fonte: Material de Camilo de taxas equivalentes

\item  Conversão do i (juros compostos):
\begin{eqnarray*}
 	i_{mensal} &=& (1+i_{anual})^{1/12} - 1 \\	
\end{eqnarray*}
 
 Fonte: Material de Camilo de taxas equivalentes

\end{enumerate}

Observacoes: 

$ ^{1} $ : Faz-se ainda um novo cálculo do pv com o valor resultante do n. Se o valor retornado for diferente, inverte-se o sinal do n.

$ ^{2} $ : O algoritmo base inicia com uma taxa de juros de 100\% e iterativamente, no máximo duas iterações mais externas trocando o sinal da taxa ou até achar a solução busca-se um novo valor de i. Internamente tenta-se acrescer uma estimativa atual de um valor gd, alterado de 0.5 ou -0.5 de acordo com certas condições, e verifica-se a proximidade do resultado dessa estimativa na função do fv em relação ao valor real do fv. Realizando-se três tentativas consecutivas de cálculo de fv que fiquem com um erro inferior a $ 1e-8 $ para-se o algoritmo, ou então tenta-se um número máximo de 400 iterações internas em estimativas do i.

\end{document}
